% !TEX root = ../main.tex

\section{Предельные теоремы теории вероятности}


\subsection{Неравенство Чебышева}

\begin{thm}[первое неравенство Чебышева]
	
\end{thm}

\begin{thm}[второе неравенство Чебышева]
	
\end{thm}

\begin{exm}
	По результатам многочисленных наблюдений, среднесуточный расход воды, в некотором населённом пункте, составляет $50000$~(л). Оценить вероятность того, что в некоторый день расход воды превысит $150000$~(л).
\end{exm}

\begin{slv}
	Пусть $X$ --- случайная величина принимающая некоторое значение, равное суточному расходу воды (л).
	\[
		X \geq 0, \quad MX = 50000
	\]
	Используя \textit{первое неравенство Чебышева} получим
	\[
		\Prob\{X \geq 150000\} \leq \frac{MX}{150000} = \frac{1}{3}
	\]
	\textbf{Ответ:} $\Prob\{X \geq 150000\} \leq 1/3$;
\end{slv}

\begin{exm}
	Пусть $X$ --- случайная величина, $MX = 1$, $\sigma = \sqrt{DX} = 0.2$. Оценить вероятности событий:
	\begin{enumerate}
		\item $\{ 0.5 \leq X < 1.5 \}$;
		\item $\{ 0.75 \leq X < 1.35 \}$;
		\item $\{ X < 2 \}$.	
	\end{enumerate}
\end{exm}

\begin{slv}[$\{ 0.5 \leq X < 1.5 \}$]\hfill\\
	Используем \textit{второе неравенство Чебышева}
	\[
		\Prob\bigl\{ | X - MX | \geq \varepsilon \bigl\} \leq \frac{DX}{\varepsilon^2};
	\]
	Приведём событие к форме, которая будет работать \textit{со вторым неравенством Чебышева}
	\begin{multline*}
		\qquad\qquad\{ 0.5 \leq X < 1.5 \} = \{ -0.5 \leq X - 1 < 0.5 \} \supseteq \\
		\supseteq \{ -0.5 < X - 1 < 0.5 \} = \bigl\{ |X - 1| < 0.5 \bigr\}\qquad\qquad
	\end{multline*}
	Строим вероятность противоположного события $\bigl\{ |X - 1| < 0.5 \bigr\}$
	\[
		\Prob \bigl\{ |X - 1| < 0.5 \bigr\} = 1 - \Prob \bigl\{ |X - 1| \geq 0.5 \bigr\}
	\]
	Рассмотрим второе неравенство Чебышева. Умножим на $-1$
	\[
		- \Prob\bigl\{ | X - MX | \geq \varepsilon \bigr\} \geq - \frac{DX}{\varepsilon^2}
	\]
	Прибавим $1$, получаем
	\begin{equation}
		1 - \Prob\bigl\{ | X - MX | \geq \varepsilon \bigr\} \geq 1 - \frac{DX}{\varepsilon^2};
	\end{equation}
	Используя данную форму, решаем поставленную задачу
	\[
		1 - \Prob \bigl\{ |X - 1| \geq 0.5 \bigr\} \geq 1 - \Biggl(\frac{0.2}{0.5}\Biggr)^2 = \frac{21}{25}
	\]
	\textbf{Ответ:} $\Prob \{ 0.5 \leq X < 1.5 \} \geq 21/25$.
\end{slv}

\begin{slv}[$\{ 0.75 \leq X < 1.55 \}$]\hfill\\
	Приводим событие к необходимой форме 
	\begin{multline*}
		\qquad\qquad\{ 0.75 \leq X < 1.35 \} = \{ -0.25 \leq X - 1 < 0.35 \} \supseteq \\
		\supseteq \{ -0.25 < X - 1 < 0.25 \} = \bigl\{ |X - 1| < 0.25 \bigr\}\qquad\qquad
	\end{multline*}
	Строим вероятность противоположного события $\bigl\{ |X - 1| < 0.25 \bigr\}$
	\[
		\Prob \bigl\{ |X - 1| < 0.25 \bigr\} = 1 - \Prob \bigl\{ |X - 1| \geq 0.25 \bigr\}
	\]
	Применяем \textit{второе неравенство Чебышева}, получаем
	\[
		1 - \Prob \bigl\{ |X - 1| \geq 0.25 \bigr\} \geq 1 - \Biggl(\frac{0.2}{0.25}\Biggr)^2 = \frac{9}{25} 
	\]
	\textbf{Ответ:} $\Prob \{ 0.75 \leq X < 1.55 \} \geq 9/25$.
\end{slv}

\begin{slv}[$\{ X < 2 \}$]\hfill\\
	Приводим событие к необходимой форме 
	\[
		\{ X < 2 \} = \{ X - 1 < 1 \} \supseteq \{ -1 < X - 1 < 1 \} = \bigl\{ |X - 1| < 1 \bigr\}
	\]
	Строим вероятность противоположного события $\bigl\{ |X - 1| < 1 \bigr\}$
	\[
		\Prob \bigl\{ |X - 1| < 1 \bigr\} = 1 - \Prob \bigl\{ |X - 1| \geq 1 \bigr\}
	\]
	Применяем \textit{второе неравенство Чебышева}, получаем
	\[
		1 - \Prob \bigl\{ |X - 1| \geq 1 \bigr\} \geq 1 - \Biggl(\frac{0.2}{1}\Biggr)^2 = 0.96
	\]
	\textbf{Ответ:} $\Prob \{ X < 2 \} \geq 0.96$.
\end{slv}

\begin{rem}
	Если использовать \textit{первое неравенство Чебышева} для последнего примера
	\[
		\Biggl[\Prob \{ X < 2 \} = 1 - \Prob \{ X < 2 \}\Biggr] \geq \Biggl[1 - \frac{MX}{2} = 1 - \frac{1}{2} = \frac{1}{2}\Biggr] 
	\]
	Таким образом, использование информации о дисперсии \textit{случайно величины} $X$ существенно уточняет оценку.
\end{rem}



\section{Закон больших чисел (ЗБЧ)}

\begin{defn}
	Последовательность $X, \dots, X_n, \dots$ \textit{удовлетворяет ЗБЧ}, если вероятность 
	\begin{equation}
		\Prob \left\{ \frac{1}{n} \sum_{i = 1}^{n} X_i - \frac{1}{n} \sum_{i = 1}^{n} m_i  \geq \varepsilon \right\} \xrightarrow[n \to \infty]{} 0, \quad \text{где}
	\end{equation}
	\begin{itemize}
		\item $m_i = MX_i$, $i \in N$
	\end{itemize}
\end{defn}

\begin{thm}[Чебышева о достаточном условие применимости ЗБЧ] Пусть
	\begin{itemize}
		\item $X_1, \dots, X_n, \dots$ --- последовательность независимых случайных величин;
		\item $\exists MX_i = m_i$, $\exists DX_i = \sigma_i^2$;
		\item $\exists C > 0 \quad \forall i \in N \quad \sigma_i^2 \leq C$ю
	\end{itemize}
	Тогда последовательность $X_1, \dots, X_n, \dots$ \textit{удовлетворяет ЗБЧ}.
\end{thm}

\begin{exm}
	Дана последовательность $X_2, X_3, \dots, X_n, \dots$ независимых случайных величин. 
	\begin{center}\begin{tabular}{|| c || c | c | c |}
		\hline
		Значения случайной величины $X_n$ & $-na$ & $0$ & $na$ \\
		\hline
		Вероятность & $1/n^2$ & $1 - 2/n^2$ & $1/n^2$ \\
		\hline
	\end{tabular}\end{center}
	Удовлетворяет ли последовательность $X_2, X_3, \dots, X_n, \dots$ закону больших чисел.
\end{exm}

\begin{slv}
	Проверим выполнимость достаточного условия (2.1 ссылка)
	\begin{align*}
		MX_n &= \sum_{i = 1}^{3} x_i\,p_i = \frac{-na}{n^2} + \frac{na}{n^2} = 0;  \\
		DX_n &= MX^2 - \cancelto{0}{(MX)^2}\;\;\; = \sum_{i = 1}^{3} x_i^2\, p_i = \frac{(-na)^2}{n^2} + \frac{(na)^2}{n^2} = 2a^2.
	\end{align*}
	\textbf{Ответ:} Удовлетворяет ЗБЧ.
\end{slv}

\begin{exm}
	Дана последовательность $X_2, X_3, \dots, X_n, \dots$ независимых случайных величин. 
	\begin{center}\begin{tabular}{|| c || c | c |}
		\hline
		Значения случайной величины $X_n$ & $-\sqrt{\ln n}$ & $\sqrt{\ln n}$ \\
		\hline
		Вероятность & $1/2$ & $1/2$ \\
		\hline
	\end{tabular}\end{center}
	Удовлетворяет ли последовательность $X_2, X_3, \dots, X_n, \dots$ закону больших чисел?
\end{exm}

\begin{slv}
	\begin{align*}
		MX_n &= \sum_{i = 1}^{2} x_i\, p_i = \frac{-\sqrt{\ln n}}{2} + \frac{\sqrt{\ln n}}{2} = 0; \\
		DX_n &= MX^2 - \cancelto{0}{(MX)^2}\;\;\; = \sum_{i = 1}^{2} x_i^2\, p_i = \frac{\ln n}{2} + \frac{\ln n}{2} = \ln n.
	\end{align*}
	$DX_n$ --- не ограничена в совокупности
	\textbf{Ответ:} Последовательность $X_2, X_3, \dots, X_n, \dots $ не удовлетворяет ЗБЧ в форме Чебышева
\end{slv}