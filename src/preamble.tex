%%% Работа с русским языком
\usepackage{cmap}					% поиск в PDF
\usepackage{mathtext} 				% русские буквы в формулах
\usepackage[T2A]{fontenc}			% кодировка
\usepackage[utf8]{inputenc}			% кодировка исходного текста
\usepackage[english, russian]{babel}	% локализация и переносы
\usepackage{color} 				 	% цветные буковки



%%% Математика
\usepackage{amsmath, amsfonts, amssymb, mathtools} % AMS
\usepackage{cancel} % \cancel зачеркивание в формулах



%%% Теоремы, Определения
\usepackage{amsthm}

\theoremstyle{definition} % "Определение"
\newtheorem{thm}{Теорема}[section]
\newtheorem{defn}{Определение}[section]
\newtheorem{exm}{Пример}[section]
\newtheorem*{rem}{Замечание}
\newtheorem*{slv}{Решение}



%%% Форматирование
% Выделение + курсив к куску текста
\newcommand{\bi}[1]{%
	\textbf{\textit{#1}}%
}



\usepackage{tikz}



%%% Поля страницы
\usepackage[top=20mm, bottom=20mm, left=30mm, right=15mm]{geometry}



%%% Вставка иллюстраций
\usepackage{graphicx}
\graphicspath{{img/}}